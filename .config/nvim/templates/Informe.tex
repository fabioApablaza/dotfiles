<+^I+>^I!comp!^I!exe!
\documentclass{article}
\usepackage[spanish]{babel}
\usepackage[utf8]{inputenc}
\usepackage{graphicx}
\usepackage{subfigure}
\usepackage{float}
\usepackage{mathtools}
\usepackage{url}
\usepackage[spanish]{babel}
\usepackage[breaklinks=true]{hyperref}
\title{Trabajo <++>}
\author{Fabio Apablaza}

\begin{document}
\titlepage
\date{<++>} %La fecha
\Large
\begin{center}


\vspace{0.7cm}
{\LARGE {\bf Universidad Nacional del Comahue}}\\
{\Large { Facultad de Informática}}\\
{\large {<++>}}\\%Nombre de la materia

\vspace{-3cm}
\mbox{\hspace{-3.5cm}\includegraphics[width=2.5cm,height=2.5cm]{Logos/unc.png}\hspace{14cm}
\includegraphics[width=2.5cm,height=2.5cm]{Logos/fai.png}}



\vspace{2cm}



\ \\
{

	\ \\
	\textbf{Trabajo <++>}\\
	<++>	%Titulo del trabajo
	\ \\

\ \\
\ \\

\ \\

\ \\

\ \\

\ \\

\ \\
\vfill
{\Large Apablaza Fabio Martin
\ \\
\texttt{fabio.apablaza@est.fi.uncoma.edu.ar}
\ \\
\texttt{FAI-2039}
}
\ \\
\ \\
{\LARGE Cursado\ 2021}
\ \\
}
\end{center}
\normalsize
\newpage
\tableofcontents

%Aquí Empieza el documento
\newpage
\section{Introducción}
<++>

\section{<++>}
\newpage
\section{Bibliografía}
\bibliographystyle{plain}
\bibliography{bibliography.bib}
\end{document}
